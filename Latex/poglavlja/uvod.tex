\section{Увод}
У данашњем, дигиталном добу, нагли пораст броја уређаја повезаних на интернет (\textit{Internet of Things - IoT}) представља један од најзначајнијих изазова за традиционалне рачунарске парадигме. Од обичних кућних апарата до сложених индустријских сензора, број \textit{IoT} уређаја расте експоненцијално, доносећи са собом потребу за ефикасним и скалабилним приступима обради података.

Претходне парадигме, попут класичних рачунарских архитектура и централизованих \textit{cloud} система суочавају се са многобројним изазовима. Уколико се знатно повећа број крајњих уређаја, количина података који се преносе преко мреже може довести до загушења мреже као и повећања латенције обраде података, што код система који захтевају обраду података у реалном времену може представљати велики проблем. У случајевима рада апликација са осетљиивм корисничким подацима, њихово пренос на \textit{cloud} платформе представља безбедносни ризик. Оно што свакако не треба занемарити је и да константна комуникација великог броја крајњих уређаја са удаљеним рачунарским центрима у неким случајевима може произвести и знатне енергетске губитке.

Како би се ови изазови превазишли , развијају се нови концепти као што су \textit{Edge, Fog} и \textit{Continuum computing} чија је основна идеја приближавање обраде и анализе података њиховим изворима. У даљим поглављима биће детаљније објашњене идеје иза сваке од горе наведене 3 парадигме, архитектуре за њихову реализацију као и неки од примера њихове употребе.