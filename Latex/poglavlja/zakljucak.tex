\section{Закључак}
 
У овом семинарском раду, истраженe су архитектуре система великих скупова података којe обухватају \textit{Edge, Fog} и \textit{Continuum computing}. Свака од ових парадигми има своје јединствене карактеристике и примене у модерном дигиталном окружењу.

\textit{Edge computing} се фокусира на обраду података крај самог извора, што доводи до смањења латенције и бољег одзива за крајње кориснике. Онo омогућава примену специфичних алгоритама и решења на уређајима на месту самих података.

\textit{Fog computing} нуди додатни слој за процесну обраду између уређаја на рубу мреже и централизованих \textit{cloud} система. Овaкав приступ побољшава скалабилност и могућност ресурсно захтевне анализе података пре њиховог слања на \textit{cloud}, што је посебно значајно у условима великог броја података и аналитички захтевних задатака.

\textit{Continuum computing} представља еволуцију ових парадигми интегришући их у кохезиван динамички систем. Он омогућава усклађивање локализованoг \textit{edge}-а, средишњег \textit{fog}-а и централизованог \textit{cloud-а}. Оваква интеграција отвара врата за унапређене могућности обраде података, побољшавање ефикасности мрежа и оптимизацију ресурса.

Захваљујући овим технологијама, предстојећи развој информационих система имаће значајне користи у областима као што су \textit{IoT}, мобилна комуникација, паметни градови, индустрија и здравствo што се већ и сад може наслутити узимајући у обзир тренутне имплементације.